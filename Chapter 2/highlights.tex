\title{Chapter 2: Multi-armed Bandits - Highlights}

\author{Jacob Taylor Cassady}

\documentclass{article}
\usepackage{textgreek}
\usepackage{amsmath, amssymb}

\begin{document}

\maketitle

\begin{itemize}
    \item The most important feature distinguishing reinforcement learning from other types of learning is that it uses training information that \emph{evaluates} the actions taken rather than \emph{instructs} by giving correct actions.
    \item Purely evaluative feedback indicates how good the action taken was, but not whether it was the best or the worst action possible.
    \item one that does not involve learning to act in more than one situation. This \emph{nonassociative} setting
    \item \emph{associative}, that is, when actions are taken in more than one situation.
\end{itemize}

\section{A k-armed Bandit Problem}
\begin{itemize}
    \item You are faced repeatedly with a choice among \emph{k} different options, or actions. After each choice you receive a numerical reward chosen from a stationary probability distribution that depends on the action you selected. Your objective is to maximize the expected total reward over some time period, for example, over 1000 action selections, or \emph{time steps}.
    \item This is the original form of the \emph{k-armed bandit problem}, so named by analogy to a slot machine, or “one-armed bandit,” except that it has k levers instead of one.
    \item Through repeated action selections you are to maximize your winnings by concentrating your actions on the best levers.
    \item Today the term “bandit problem” is sometimes used for a generalization of the problem described above, but in this book we use it to refer just to this simple case.
    \item In our k-armed bandit problem, each of the k actions has an expected or mean reward given that that action is selected; let us call this the \emph{value} of that action. We denote the action selected on time step t as $A_t$, and the corresponding reward as $R_t$. The value then of an arbitrary action a, denoted $q_*(a)$, is the expected reward given that \emph{a} is selected: $$ q_*(a) = \mathbb{E}[R_t | A_t = a] $$
    \item We assume that you do not know the action values with certainty, although you may have estimates. We denote the estimated value of action \emph{a} at time step \emph{t} as $Q_t(a)$. We would like $Q_t(a)$ to be close to $q_*(a)$.
    \item If you maintain estimates of the action values, then at any time step there is at least one action whose estimated value is greatest. We call these the \emph{greedy} actions. When you select one of these actions, we say that you are \emph{exploiting} your current knowledge of the values of the actions. If instead you select one of the nongreedy actions, then we say you are \emph{exploring}, because this enables you to improve your estimate of the nongreedy action's value.
    \item Reward is lower in the short run, during exploration, but higher in the long run because after you have discovered the better actions, you can exploit them many times.
\end{itemize}

\section{Action-value Methods}
\begin{itemize}
    \item .
\end{itemize}

\section{The 10-armed Testbed}
\begin{itemize}
    \item .
\end{itemize}

\section{Incremental Implementation}
\begin{itemize}
    \item .
\end{itemize}

\section{Tracking a Nonstationary Problem}
\begin{itemize}
    \item .
\end{itemize}

\section{Optimistic Initial Values}
\begin{itemize}
    \item .
\end{itemize}

\section{Upper-Confidence-Bound Action Selection}
\begin{itemize}
    \item .
\end{itemize}

\section{Gradient Bandit Algorithms}
\begin{itemize}
    \item .
\end{itemize}

\section{Associative Search (Contextual Bandits)}
\begin{itemize}
    \item .
\end{itemize}

\section{Summary}
\begin{itemize}
    \item .
\end{itemize}

\end{document}
