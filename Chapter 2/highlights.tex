\title{Chapter 2: Multi-armed Bandits - Highlights}

\author{Jacob Taylor Cassady}

\documentclass{article}
\usepackage{textgreek}
\usepackage{bbold}
\usepackage{dsfont}
\usepackage{amsmath, amssymb}

\begin{document}

\maketitle

\begin{itemize}
    \item The most important feature distinguishing reinforcement learning from other types of learning is that it uses training information that \emph{evaluates} the actions taken rather than \emph{instructs} by giving correct actions.
    \item Purely evaluative feedback indicates how good the action taken was, but not whether it was the best or the worst action possible.
    \item one that does not involve learning to act in more than one situation. This \emph{nonassociative} setting
    \item \emph{associative}, that is, when actions are taken in more than one situation.
\end{itemize}

\section{A k-armed Bandit Problem}
\begin{itemize}
    \item You are faced repeatedly with a choice among \emph{k} different options, or actions. After each choice you receive a numerical reward chosen from a stationary probability distribution that depends on the action you selected. Your objective is to maximize the expected total reward over some time period, for example, over 1000 action selections, or \emph{time steps}.
    \item This is the original form of the \emph{k-armed bandit problem}, so named by analogy to a slot machine, or “one-armed bandit,” except that it has k levers instead of one.
    \item Through repeated action selections you are to maximize your winnings by concentrating your actions on the best levers.
    \item Today the term “bandit problem” is sometimes used for a generalization of the problem described above, but in this book we use it to refer just to this simple case.
    \item In our k-armed bandit problem, each of the k actions has an expected or mean reward given that action is selected; let us call this the \emph{value} of that action. We denote the action selected on time step t as $A_t$, and the corresponding reward as $R_t$. The value then of an arbitrary action a, denoted $q_*(a)$, is the expected reward given that \emph{a} is selected: $$ q_*(a) \doteq \mathds{E}[R_t | A_t = a] $$
    \item We assume that you do not know the action values with certainty, although you may have estimates. We denote the estimated value of action \emph{a} at time step \emph{t} as $Q_t(a)$. We would like $Q_t(a)$ to be close to $q_*(a)$.
    \item If you maintain estimates of the action values, then at any time step there is at least one action whose estimated value is greatest. We call these the \emph{greedy} actions. When you select one of these actions, we say that you are \emph{exploiting} your current knowledge of the values of the actions. If instead you select one of the nongreedy actions, then we say you are \emph{exploring}, because this enables you to improve your estimate of the nongreedy action's value.
    \item Reward is lower in the short run, during exploration, but higher in the long run because after you have discovered the better actions, you can exploit them many times.
\end{itemize}

\section{Action-value Methods}
\begin{itemize}
    \item methods for estimating the values of actions and for using the estimates to make action selection decisions, 
    which we collectively call \emph{action-value methods}.
    \item $$ Q_t(a) \doteq \frac{\text{sum of rewards when \emph{a} taken prior to \emph{t}}}{\text{number of times \emph{a} taken prior to \emph{t}}} = \frac{\sum_{i=1}^{t-1} R_i \cdotp \mathbb{1}_{A_i=a}}{\sum_{i=1}^{t-1}\mathbb{1}_{A_i=a}} \thickspace \thickspace \thickspace \text{(2.1)} $$
    where $ \mathbb{1}_{predicate} $ denotes the random variable that is 1 if \emph{predicate} is true and 0 if it is not.
    If the denominator is zero, then we instead define $ Q_t(a) $ as some default value, such as
    0. As the denominator goes to infinity, by the law of large numbers, $ Q_t(a) $ converges to
    $ q_*(a) $. We call this the \emph{sample-average} method for estimating action values because each
    estimate is an average of the sample of relevant rewards.
    \item The simplest action selection rule is to select one of the actions with the highest
    estimated value, that is, one of the greedy actions as defined in the previous section.
    \item We write this greedy action selection method as
    $$ A_t \doteq \underset{a}{argmax} \thickspace Q_t(a) $$
    \item A simple alternative is to behave greedily most of the time, but every once in a while, 
    say with small probability $ \epsilon $, instead select randomly from among all the actions with equal 
    probability, independently of the action-value estimates. We call methods using this near-greedy action selection rule
    $ \epsilon $-greedy methods.
    \item An advantage of these methods is that, in the limit as the number of
    steps increases, every action will be sampled an infinite number of times, thus ensuring
    that all the $ Q_t(a) $ converge to $ q_*(a) $. This of course implies that the probability of selecting
    the optimal action converges to greater than 1-\textepsilon, that is, to near certainty.
\end{itemize}

\section{The 10-armed Testbed}
\begin{itemize}
    \item a set of 2000 randomly generated k-armed bandit problems with k = 10. For each bandit
    problem, the action values, $q_*(a)$, a = 1, . . . , 10,
    were selected according to a normal (Gaussian) distribution with mean 0 and variance 1.
    Then, when a learning method applied to that problem selected action $A_t$ at time step t,
    the actual reward, $R_t$, was selected from a normal distribution with mean $q_*(A_t)$ and
    variance 1. We call this suite of test tasks the \emph{10-armed testbed}.
    \item The greedy method improved slightly faster than the
    other methods at the very beginning, but then leveled off at a lower level.
    \item The greedy method performed significantly worse in the long run because it
    often got stuck performing suboptimal actions.
    \item The \textepsilon-greedy methods eventually performed better because they continued
    to explore and to improve their chances of recognizing the optimal action. The \textepsilon = 0.1
    method explored more, and usually found the optimal action earlier, but it never selected
    that action more than 91\% of the time. The \textepsilon = 0.01 method improved more slowly, but
    eventually would perform better than the \textepsilon = 0.1 method on both performance measures
    shown in the figure. It is also possible to reduce \textepsilon \thickspace over time to try to get the best of both
    high and low values.
    \item suppose the reward variance had been larger, say 10 instead of 1. With noisier rewards
    it takes more exploration to find the optimal action, and \textepsilon-greedy methods should fare
    even better relative to the greedy method. On the other hand, if the reward variances
    were zero, then the greedy method would know the true value of each action after trying
    it once. In this case the greedy method might actually perform best because it would
    soon find the optimal action and then never explore.
    \item Even if the underlying task is stationary and deterministic, the learner [typically] faces a set of banditlike decision 
    tasks each of which changes over time as learning proceeds and the agent's decision-making policy changes.
\end{itemize}

\section{Incremental Implementation}
\begin{itemize}
    \item Incremental sample averages estimate 
    $$ Q_{n+1} = Q_n + \frac{1}{n}[R_n - Q_n] $$
\end{itemize}

\section{Tracking a Nonstationary Problem}
\begin{itemize}
    \item .
\end{itemize}

\section{Optimistic Initial Values}
\begin{itemize}
    \item .
\end{itemize}

\section{Upper-Confidence-Bound Action Selection}
\begin{itemize}
    \item .
\end{itemize}

\section{Gradient Bandit Algorithms}
\begin{itemize}
    \item .
\end{itemize}

\section{Associative Search (Contextual Bandits)}
\begin{itemize}
    \item .
\end{itemize}

\section{Summary}
\begin{itemize}
    \item .
\end{itemize}

\end{document}
